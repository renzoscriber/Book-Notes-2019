\documentclass[a4paper]{article}
% Preamble
\usepackage[utf8]{inputenc}
\usepackage{fullpage}
\usepackage[english]{babel}
\usepackage{color}
\usepackage{amsmath}
\usepackage{url}
\usepackage{standalone}
\usepackage{parskip}
\usepackage{graphicx}
\usepackage{caption}
\usepackage{subcaption}
\usepackage{natbib}
\usepackage{amsfonts}
\usepackage{hyperref}

\title{On Gravity}
\author{Renzo Scriber}
\date{Fall 2019}

\begin{document}

\maketitle

\section*{Source}
Title: On Gravity
Author: A. Zee
\section*{Part I}
    \subsection*{Chapter 1}
    \subsection*{Chapter 2}
    \subsection*{Chapter 3}
    \subsection*{Chapter 4: From water waves to gravity waves}
        \begin{quote}
            The equation for fluid flow has not been solved in its full generality to this very day. In fact, a million dollars is yours if you can solve it. (p.32)
        \end{quote}
\section*{Part II}
    \subsection*{Chapter 5}
    \subsection*{Chapter 6}
    \subsection*{Chapter 7}
    \subsection*{Chapter 8}
    \subsection*{Chapter 9}
\section*{Part III}
    \subsection*{Chapter 10: Getting the best possible deal}
        \subsubsection*{Fermat's least time principle for light}
            \begin{itemize}
                \item \textbf{Fermat's least time prinple for light} = Light has a tendency to take the \textit{path of least time}; 
                \item In other words, light prioritizes time over space
            \end{itemize}
        \subsubsection*{Material particles}
            \begin{itemize}
                \item Matter behaves differently from light (e.g., it does not travel at a constant speed)
                \item Like Fermat's least time principle for light, something is minimized for material particles. But what is it?
                \item For material particles, a fundamental quantity called the \textit{action} can be used to formulate an analog to Fermat's least time principle for light
            \end{itemize}
        \subsubsection*{Choice of history: The action principle}
            \begin{itemize}
                \item The Lagrangian of a particle at any moment in time is equal to the particle's potential energy subtracted from its kinetic energy
                \item Notes from \href{https://www.khanacademy.org/math/multivariable-calculus/applications-of-multivariable-derivatives#optimizing-multivariable-functions-videos}{Khan Academy - Multivariable Calculus - Applications of Multivariable Derivatives - Optimizing Multivariable Functions}
                    \begin{itemize}
                        \item \textbf{Lagrange multipliers} are proportinality constants that allow us to set the gradient of one function equal to the gradient of another function
                        \item The Lagrangian is a concept highly related to the Lagrange multiplier and both involve constraint optimization problems
                        \item Importantly, we can find the $f(x,y)=c_{\mathrm{max}}$ contour line that is tangent ot your constraint curve $g(x,y)=b$ by setting $\nabla f(x,y) = \lambda \nabla g(x,y)$ where $\lambda$ dentoes the Lagrange multiplier
                        \item This is because $\nabla f(x,y)$ and $\nabla g(x,y)$ both point in the same direction at the point of tangency between $f(x,y)=c_{\mathrm{max}}$ and  $g(x,y)=b$ BUT their magnitudes differ by a factor of $\lambda$
                    \end{itemize}
                \item The \textbf{action} of a material particle over some $\delta t$ is equal to the integral of its Lagrangians over time
                \item The \textbf{action principle} states that the material particle picks a path that either maximizes or minimizes the action
            \end{itemize}
        \subsubsection*{Brevity is the soul of wit}
            \begin{quote}
                The fundamental interactions we know about, the strong, weak, electromagnetic, and gravitational, can all be described by the action principle
            \end{quote}
            \begin{itemize}
                \item We don't know why the fundamental interactions follow the action principle 
                \item Note: There are many equations of motion that do not follow the action principle!
            \end{itemize}
    \subsection*{Chapter 11}
    \subsection*{Chapter 12}
    \subsection*{Chapter 13}
    \subsection*{Chapter 14}
\section*{Part IV}
    \subsection*{Chapter 15}
    \subsection*{Chapter 16}
    \subsection*{Chapter 17}
    \subsection*{Chapter 18}
    \subsection*{Chapter 19}
\section*{Appendix}
\end{document}