\documentclass[a4paper]{article}
% Preamble
\usepackage[utf8]{inputenc}
\usepackage{fullpage}
\usepackage[english]{babel}
\usepackage{color}
\usepackage{amsmath}
\usepackage{url}
\usepackage{standalone}
\usepackage{parskip}
\usepackage{graphicx}
\usepackage{caption}
\usepackage{subcaption}
\usepackage{natbib}
\usepackage{amsfonts}
\usepackage{hyperref}

\title{Introductory Quantum Physics and Relativity}
\author{Renzo Scriber}
\date{Fall 2019}

\begin{document}

\maketitle

\section*{Source}
Title: Introductory Quantum Physics and Relativity
\\
\\
Author: Jacob Dunningham and Vlatko Vedral
\section{Introduction}
Quantum physics and relativity were developed to deal with the failures of Newtonian physics
    \begin{itemize}
        \item Newtonian physics fails for...
            \begin{enumerate}
                \item Small particles
                \item Large speeds
            \end{enumerate}
        \item Goal of this book: To give the reader an appreciation of the important role that quantum physics and relativity play in modern physics
    \end{itemize}
Principles laid down by Socrates:
    \begin{enumerate}
        \item Knowledge can be pursued and is worth pursuing
        \item The search for knowledge is a cooperative enterprise
        \item A question is a form of education that draws out what is in a person rather than imposing on him a view from outside
        \item Knowledge must be pursued with a ruthless intellectual honesty
    \end{enumerate}
Karl Popper: Realized that physics and science in general makes progress via the method of conjectures and refutations 
    \begin{quote}
        "No number of sightings of white swans can ever prove the hypothesis that all swans are white, but observing a single black swan can and does completely invalidate it." - David Hume
    \end{quote}
    \begin{itemize}
        \item The two "black swans" of physics are quantum mechanics and relativity
    \end{itemize}
Newtonian physics 
    \begin{itemize}
        \item Implies that Newton's laws of motion fully determine the future behavior of a system/object given its position, velocity, mass, and all the forces acting on the object 
    \end{itemize}
    \begin{quote}
        "The compression of all facts into a few simple laws of nature is precisely the point of physics 
    \end{quote}
Boolean logic
    \begin{itemize}
        \item Classical systems/objects also obey the laws of Boolean logic
        \item \textbf{Excluded middle} = an object either exists in a particular state or doesn't exist in that particular state - there is no third option
            \begin{itemize}
                \item This is the key law of Boolean logic
            \end{itemize}
        \item Q: Is logic something that is fundamentally different from physics?
        \item \textbf{The law of commutativity of propositions} = the order in which a sequence of propositions is stated does not affect the meaning of the overall proposition
            \begin{itemize}
                \item This is another key feature of Boolean logic
            \end{itemize}
    \end{itemize}
The microscopic world does not obey Boolean logic
    \begin{itemize}
        \item E.g., an electron can simultaneously exist in many positions or have many velocities (violation of the excluded middle)
        \item E.g., "An electron is positioned here at time 0 and then has a velocity at time 1" has a different meaning from that of "an electron has a velocity at time 0 and is positioned here at time 1" (violation of the law of commutativity)
    \end{itemize}
\textbf{Mach-Zehnder interferometer} = commonly used to illustrate the non-Boolean nature of photons
    \begin{itemize}
        \item Demonstrates the wave-nature of photons (i.e., interference patterns emerge)
        \item We use states to 
    \end{itemize}
\\
\\
Notation: Let $\psi_a \rightarrow \psi_b + \psi_c$ denote the case where a system goes from some state $a$ into an equal superposition of states $b$ and $c$
\section{Old Quantum Theory}
Max Planck = first to suggest that energy is quantized 
\subsection{Black Body Radiation}
Around 1859 (same year Darwin published his \textit{Origin of Species}), the German experimental physicist, Krichhoff conjectured that determination the energy spectrum for a given black body was the "holy grail" of physics 
\\
\\
Two basic ways in which heat propagates through a given medium
    \begin{itemize}
        \item Conduction = the rate of temperature change depends on a temperature gradient; goverened by a diffusion equation
            \begin{equation}
                \frac{dT}{dt} = -\alpha \frac{d^2}{dx^2}T
            \end{equation}
            \begin{itemize}
                \item $\alpha$ is a constant that depends on the medium
                \item Once a medium has uniformly distrubuted temperature, $\Dot{T}=0$
            \end{itemize}
        \item Radiation = independent of a temperature gradient
            \begin{itemize}
                \item Radiation can and does take place at a constant temperature (which can make it more difficult to study)
                \item Consider the following two cases:
                    \begin{enumerate}
                        \item White body = an object that reflects all incident EM radiation
                        \item Black body = an object that absorbs all incident EM radiation
                            \begin{itemize}
                                \item Approximated examples: The Sun and the Earth
                                \item Questions regarding the radiation leaving a black body:
                                    \begin{itemize}
                                        \item What kind of properties woulod the radiation have?
                                        \item Would would we observe if we were inside the black body?
                                    \end{itemize}
                                \item Classical prediction: 
                                    \begin{itemize}
                                        \item Every atom in a black body will emit radiation at all possible frequencies
                                        \item At temperature $T$, the energy of each frequency is equal to $kT$, where $k$ denotes Boltzmann's constant
                                        \item Predicts that a black body will emit an infinite amount of radiation
                                    \end{itemize}
                                 \item The classical prediction for black body radiation turned out to be wrong and became known as the \textbf{ultraviolet catastrophe}
                                
                            \end{itemize}
                    \end{enumerate}
            \end{itemize}
    \end{itemize}
Planck's guess equation for the energy density of a black body:
    \begin{equation}
        E = hf\frac{1}{e^{hf/kT}-1}
    \end{equation}
    \begin{itemize}
        \item Variables
            \begin{itemize}
                \item $h \approx 6.663 \cdot 10^{-34} \mathrm{Js}$ denotes Planck's constant
                \item $f$ denotes the frequency of the radiation 
                \item $k$ denotes Boltzmann's constant
                \item $T$ denotes the temperature of the black body
                \item Classical limit: $T$ is sufficiently high so that $e^{hf/kT} \approx 1 + hf/kT$ and thus, $E \approx kT$
            \end{itemize}
        \item Gives the \textit{average} energy emitted by a black body at a particular frequency
    \end{itemize}
Density of states (per unit volume) for a black body:
    \begin{equation}
        g(f) = \frac{8\pi}{c^3}f^2
    \end{equation}
\section{Quantum Mechanics}
Postulates of Quantum Mechanics
\begin{enumerate}
    \item States of physical systems are represented by complex functions called wave functions
    \item Observables of physical systems, that is, things that we can measure, such as the position or momentum of a particle, are represented by Hermitian operators
    \item \textbf{Born measurement postulate}: When we make a measurement of the system, we probabilistically obtain the answer according to the modulus square of the wave function
    \item The system, when not measured, evolves according to the Schrodinger equation
\end{enumerate}
\section{Applications of Quantum Mechanics}
\section{Schrodinger Equation in Three Dimensions}
\section{Spin and Statistics}
\section{Atoms, Molecules and Lasers}
\section{Formal Structure of Quantum Mechanics}
\section{Second Revolution: Relativity}
\section{Fine Structure of the Hydrogen Atom}
\section{Relativistic Quantum Mechanics}
\section{Quantum Entanglement}
\section{Solutions}
\end{document}