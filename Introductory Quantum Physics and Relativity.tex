\documentclass[a4paper]{article}
% Preamble
\usepackage[utf8]{inputenc}
\usepackage{fullpage}
\usepackage[english]{babel}
\usepackage{color}
\usepackage{amsmath}
\usepackage{url}
\usepackage{standalone}
\usepackage{parskip}
\usepackage{graphicx}
\usepackage{caption}
\usepackage{subcaption}
\usepackage{natbib}
\usepackage{amsfonts}
\usepackage{hyperref}

\title{Introductory Quantum Physics and Relativity}
\author{Renzo Scriber}
\date{Fall 2019}

\begin{document}

\maketitle

\section*{Source}
Title: Introductory Quantum Physics and Relativity
\\
\\
Author: Jacob Dunningham and Vlatko Vedral
\section{Introduction}
Quantum physics and relativity were developed to deal with the failures of Newtonian physics
    \begin{itemize}
        \item Newtonian physics fails for...
            \begin{enumerate}
                \item Small particles
                \item Large speeds
            \end{enumerate}
        \item Goal of this book: To give the reader an appreciation of the important role that quantum physics and relativity play in modern physics
    \end{itemize}
Principles laid down by Socrates:
    \begin{enumerate}
        \item Knowledge can be pursued and is worth pursuing
        \item The search for knowledge is a cooperative enterprise
        \item A question is a form of education that draws out what is in a person rather than imposing on him a view from outside
        \item Knowledge must be pursued with a ruthless intellectual honesty
    \end{enumerate}
Karl Popper: Realized that physics and science in general makes progress via the method of conjectures and refutations 
    \begin{quote}
        "No number of sightings of white swans can ever prove the hypothesis that all swans are white, but observing a single black swan can and does completely invalidate it." - David Hume
    \end{quote}
    \begin{itemize}
        \item The two "black swans" of physics are quantum mechanics and relativity
    \end{itemize}
Newtonian physics 
    \begin{itemize}
        \item Implies that Newton's laws of motion fully determine the future behavior of a system/object given its position, velocity, mass, and all the forces acting on the object 
    \end{itemize}
    \begin{quote}
        "The compression of all facts into a few simple laws of nature is precisely the point of physics 
    \end{quote}
Boolean logic
    \begin{itemize}
        \item Classical systems/objects also obey the laws of Boolean logic
        \item \textbf{Excluded middle} = an object either exists in a particular state or doesn't exist in that particular state - there is no third option
            \begin{itemize}
                \item This is the key law of Boolean logic
            \end{itemize}
        \item Q: Is logic something that is fundamentally different from physics?
        \item \textbf{The law of commutativity of propositions} = the order in which a sequence of propositions is stated does not affect the meaning of the overall proposition
            \begin{itemize}
                \item This is another key feature of Boolean logic
            \end{itemize}
    \end{itemize}
The microscopic world does not obey Boolean logic
    \begin{itemize}
        \item E.g., an electron can simultaneously exist in many positions or have many velocities (violation of the excluded middle)
        \item E.g., "An electron is positioned here at time 0 and then has a velocity at time 1" has a different meaning from that of "an electron has a velocity at time 0 and is positioned here at time 1" (violation of the law of commutativity)
    \end{itemize}
\textbf{Mach-Zehnder interferometer} = commonly used to illustrate the non-Boolean nature of photons
    \begin{itemize}
        \item Demonstrates the wave-nature of photons (i.e., interference patterns emerge)
        \item We use states to 
    \end{itemize}
\\
\\
Notation: Let $\psi_a \rightarrow \psi_b + \psi_c$ denote the case where a system goes from some state $a$ into an equal superposition of states $b$ and $c$
\section{Old Quantum Theory}
\section{Quantum Mechanics}
\section{Applications of Quantum Mechanics}
\section{Schrodinger Equation in Three Dimensions}
\section{Spin and Statistics}
\section{Atoms, Molecules and Lasers}
\section{Formal Structure of Quantum Mechanics}
\section{Second Revolution: Relativity}
\section{Fine Structure of the Hydrogen Atom}
\section{Relativistic Quantum Mechanics}
\section{Quantum Entanglement}
\section{Solutions}
\end{document}